\chapter*{Introducere} 
\addcontentsline{toc}{chapter}{Introducere}

\begin{sloppypar}
În acest capitol introductiv voi face o scurtă prezentare a limbajului Dafny, urmată de câteva puncte esențiale despre paradigma programării dinamice și câteva detalii despre Problema Rucsacului.

\subsection*{Limbajul de programare Dafny}
Dafny este un limbaj de programare imperativ și funcțional, ce oferă suport pentru programarea orientată pe obiecte și ce este destinat verificării formale a corectudinii programelor. \cite{enwiki:1265861302} A fost creat pentru a ajuta programatorii să scrie cod care este corect din punct de vedere al specificațiilor. Un lucru foarte interesant de știut despre acest limbaj este faptul că a fost conceput astfel încât permite verificarea codului încă din faza de dezvoltare, înainte de a rula codul. Datorită faptului că verificatorul Dafny este rulat ca parte a compilatorului, orice eroare matematică sau logică va fi semnalată către programator care va trebui să ajute verificatorul prin ajustarea specificațiilor sau a logicii. \cite{leino2021dafny}

\subsection*{Programarea dinamică}
Programarea dinamică este metodă generală folosită pentru a proiecta o clasă de algoritmi ce rezolvă probleme cu proprietăți similare. \cite{Algorithm-Design} Ea se bazează pe conceptul de suprapunere al subproblemelor, însemnând faptul că o problemă poate fi "spartă" în mai multe subprobleme mai mici care sunt mai ușor de rezolvat și care se repetă pe parcursul execuției algoritmului. 
Soluțiile acestor subprobleme sunt stocate astfel încât să nu fie necesară recalcularea lor, reutilizând astfel soluțiile deja calculate pentru a optimiza procesul de calcul. \cite{DP-javatpoint} \par 
Principalele abordări prin care soluțiile sunt stocate când folosim programarea dinamică sunt:
\begin{itemize}
     \item \textbf{Memoizarea (Top-Down)} este abordarea recursivă, în care rezultatele sunt salvate într-o structură de date de unde vor fi accesate ulterior când este nevoie de rezultatul deja calculat;
     \item \textbf{Tabelizarea (Bottom-Up)} este abordarea în care se pornește de la calcularea celor mai mici subprobleme, construind treptat soluția finală din soluțiile subproblemelor mai mici. \cite{DP-GG}
\end{itemize} \par
În general, paradigma programării dinamice este folosită pentru a rezolva problemele de optimizare care au ca obiectiv obținerea celei mai bune soluții într-un cadru de optimizare, fie pentru minimizare, fie pentru maximizare. \par


\subsection*{Ce este Problema Rucsacului?}
Problema rucsacului este o problemă de optimizare. Presupunând că avem un rucsac care poate susține o anumită greutate maximă numită capacitate, trebuie să alegem un subset de obiecte dintr-o mulțime astfel încât valoarea totală a acestora este maximă, iar greutatea totală a obiectelor nu depășește capacitatea rucsacului. Cele mai cunoscute variațiuni ale acestei probleme sunt varianta continuă și varianta discretă. \par 
În varianta continuă, obiectele pot fi fracționate în mai multe bucăți, permițând alegerea parțială a unui obiect. Pentru această variantă a problemei, algoritmii greedy sunt mai potriviți datorită deciziilor bazate pe un criteriu local precum raport valoare/greutate, greutate mică sau câștig mare a obiectului, astfel eliminând riscul de a pierde combinații de obiecte. \par
În varianta discretă fiecare obiect poate fi ales o singură dată sau deloc. Pentru varianta discretă abordările greedy nu pot garanta că la finalul execuției algoritmului soluția obținută este optimă. \cite{Algorithm-Design} Acest lucru este datorat naturii acestor algoritmi de a alege varianta cea mai profitabilă în momentul curent și ignoră alte combinații de obiecte care individual au un profit mai mic, dar împreună ar fi mult mai profitabile, de aceea acest tip de probleme necesită o abordare mai complexă precum programarea dinamică. \\

\subsection*{Formularea problemei}
Pentru a înțelege mai bine ideea problemei și cum funcționează algoritmul vom considera ca exemplu următoarea instanță a problemei:
\begin{textbox}
\textbf{Date de intrare:} \par
$\bullet$ $n = 4$, unde $n$ este numărul total de obiecte pe care le avem la dispoziție; \par
$\bullet$ $c = 8$, unde $c$ reprezintă capacitatea totală a rucsacului; \par
$\bullet$ $gains = [1, 2, 5, 6]$, unde $gains$ reprezintă un vector de lungime $n$ cu profitul pe care îl aduce fiecare obiect; \par
$\bullet$ $weights = [2, 3, 4, 5]$, unde $weights$ reprezintă un vector de lungime $n$ cu greutatea fiecărui obiect.
\end{textbox}

\renewcommand{\labelitemi}{$\rightarrow$}

Scopul algoritmului este de a alege obiectele pentru a maximiza profitul obținut, respectând în același timp constrângerea de a nu depăși capacitatea rucsacului. Astfel, la fiecare pas algoritmul va trebui sa aleagă una din cele două opțiuni disponibile: fie obiectul va fi adăugat în rucsac integral, iar greutatea și profitul acestuia vor fi adăugate în soluție, fie obiectul nu va fi inclus în rucsac. \par
Voi explica următoarele notații care se aplică și în cadrul celorlalte capitole ale lucrării:
\begin{itemize}
    \item $i$ este folosit pentru a memora indexul obiectului curent;
    \item $j$ este folosit pentru a memora valoarea curentă a capacității parțiale a rucsacului;
    \item tuplul $(i, j)$, unde $0 \le i < n$ și $0 \le j < c$ reprezintă o subproblemă în care considerăm primele $i$ obiecte și o capacitate parțială $j$ a rucsacului. 
\end{itemize}  \par

\renewcommand{\labelitemi}{$\bullet$}

În continuare, voi detalia o schiță a algoritmului implementat de către mine care poate fi folosită pentru a rezolva varianta discretă a problemei rucsacului și pentru a obține profitul maxim pentru orice instanță a problemei:

\begin{tcolorbox}[mystyle]
\begin{enumerate}[left=0.2em]
    \item \textbf{Inițializarea:}
    \begin{itemize}
        \item Declarăm o matrice $ \textit{profits} $ de dimensiune $ (n + 1) \times (c + 1) $, unde fiecare celulă $ \textit{profits}[i][j] $ reprezintă profitul maxim posibil obținut pentru o subproblemă în care sunt considerate disponibile maxim $i$ obiecte, iar rucsacul poate susține o capacitate maximă $j$;
        \item Inițializăm prima linie din matrice cu 0. Acestea reprezintă cazurile în care pentru orice capacitate de la 0 până la $j$ vom considera că nu avem obiecte disponibile, deci profitul maxim este 0. 
    \end{itemize}
    \item \textbf{Procesul iterativ:}
    Pentru fiecare obiect $i$, unde $1 \le i < n + 1$ și fiecare capacitate $j$, unde $0 \le j < c + 1$:
    \begin{enumerate}
        \item Capacitatea rucsacului este 0, deci profitul maxim posibil este de asemenea 0.
        \item Obiectul nu poate fi inclus \par
            $\bullet$ greutatea obiectului curent $\textit{weights}[i - 1]$ depășește capacitatea $j$, atunci profitul pentru pasul curent rămâne același ca pentru $i - 1$ obiecte: $\textit{profits}[i][j] = \textit{profits}[i - 1][j]$.
        \item Ștind că greutatea obiectului curent $\textit{weights}[i - 1]$ nu depășește capacitatea $j$, obiectul poate fi inclus, dar: \par
        $\bullet$ dacă se obține un profit mai bun decât cel anterior prin includerea obiectului atunci adăugăm câștigul obiectului curent $\textit{gains}[i - 1]$ la valoarea obținută pentru capacitatea rămasă $j - \textit{weights}[i - 1]$: $\textit{profits}[i][j] = \textit{profits}[i - 1][j = \textit{weights}[i - 1]] + \textit{gains}[i - 1]$. \par
        $\bullet$ dacă se obține un profit cel mult la fel de bun ca profitul anterior obținut pentru aceeași capacitate $j$ prin adăugarea obiectului atunci se va considera profitul maxim obținut fără acest obiect: $\textit{profits}[i][j] = \textit{profits}[i - 1][j]$.
    \end{enumerate}
    \item \textbf{Rezultatul:}
    La finalul execuției algoritmului, profitul maxim care se poate obține utilizând toate obiectele pe care le avem la dispoziție pentru un rucsac de capacitate maximă $c$ se află în $\textit{profits}[n][c]$.
\end{enumerate}
\end{tcolorbox}

Pentru a stoca soluțiile, algoritmul folosește abordarea Bottom-Up în care punctul de start îl reprezintă cele mai mici subprobleme, mai exact cazurile de bază. După cum am văzut în schița algoritmului prezentat anterior, adeseori pe parcusul execuției vom avea nevoie de profitul calculat în pasul anterior obținut pentru $i - 1$ obiecte având aceeași capacitate $j$. Aici intervine proprietatea \textbf{problemelor suprapuse} caracteristică programării dinamice, în care avem nevoie de soluția unei subprobleme care a fost deja rezolvată, evitând astfel recalcularea acesteia. \par
De asemenea, rezolvarea problemei inițiale pentru un rucsac de capacitate $c$ cu $n$ obiecte poate fi construită din rezultatele obținute pentru subproblemele anterioare la care este adăugat profitul obiectului în funcție de capacitatea care permite includerea acestuia. De menționat este că la fiecare pas al procesului iterativ, mereu se va considera cel mai bun rezultat care se poate obține bazat pe două posibile decizii: \par
$\bullet$ Dacă obiectul curent este inclus în soluție, atunci rezultatul va fi calculat din profitul obiectului la care se adaugă soluția subproblemei pentru greutatea rămasă $j - \textit{weights}[i - 1]$, soluție despre care știm că este optimă pentru $i$ obiecte si capacitate $j - \textit{weights}[i - 1]$; \par
$\bullet$ Altfel dacă obiectul nu este inclus, soluția va rămâne aceeași ca cea pentru $i - 1$ obiecte și capacitate $j$, soluție despre care știm că este optimă pentru $i - 1$ obiecte si capacitate $j$. \par
Aici intervine cea de-a doua proprietate specifică programării dinamice, numită proprietatea de \textbf{substructură optimă} care constă în faptul că soluția optimă a unei probleme poate fi construită din soluțiile optime ale subproblemelor, subsoluții care sunt optime pentru subproblemele respective.
\\ \par
Pentru a obține profitul maxim pentru datele de intrare de mai sus putem aplica algoritmul descris și începem prin a stabili cazul de bază: pornim de la $i = 0$ (nu avem niciun obiect la dispoziție) și trecem prin fiecare valoare posibilă a capacității $j$. Pentru fiecare subproblemă $(0, j)$, unde $0 \le j < c $, soluția optimă este cea de profit maxim 0:

\resulttable{0}{
    0 & 0 & 0 & 0 & 0 & 0 & 0 & 0 & 0 & 0 \\
    }

Algoritmul continuă prin parcurgerea fiecărui obiect și fiecare valoare posibilă a greutății rucsacului. Pentru $i = 1$, profitul maxim care se poate obține este 1 deoarece avem un singur obiect la dispoziție indiferent de valoarea capacității rucsacului: 

\resulttable{1}{
    0 & 0 & 0 & 0 & 0 & 0 & 0 & 0 & 0 & 0 \\
    1 & 0 & 0 & 1 & 1 & 1 & 1 & 1 & 1 & 1 \\
    }

Pentru $i = 2$ considerăm doar primele două obiecte și iterăm prin valorile posibile ale capacității: \par
\begin{itemize}
\item Pentru subproblema $(2,2)$ nu putem adăuga decât primul obiect, iar profitul optim rămâne 1; 
\item Pentru subproblema $(2, 3)$ putem adăuga al doilea obiect deoarece aduce un profit mai bun decât cel anterior $\textit{profits}[i - 1][j]$ care este 1, iar restricția de a nu depăși capacitatea ruscacului este respectată; 
\item Pentru subproblema $(2, 5)$ putem adăuga ambele obiecte în rucsac, iar profitul este calculat adunând câștigul obiectului cu profitul corespunzător capacității rămase după scăderea greutății obiectului, adică rezultatul subproblemei($i - 1, j - \textit{weights}[i - 1]$) care este 1: 
\end{itemize}

\resulttable{2}{
    0 & 0 & 0 & 0 & 0 & 0 & 0 & 0 & 0 & 0 \\
    1 & 0 & 0 & 1 & 1 & 1 & 1 & 1 & 1 & 1 \\
    2 & 0 & 0 & 1 & 2 & 2 & 3 & 3 & 3 & 3 \\
}

Aplicând aceeași logică pentru $i = 3$ vom obține următoarele rezultate: \par
$\bullet$ Pentru subproblema $(3, 4)$ putem adăuga al treilea obiect deoarece aduce un profit mai bun decât rezultatul suproblemei anterioare cu aceeași capacitate $(2, 4)$; \par
$\bullet$ Pentru subproblema $(3, 5)$ obținem un profit mai bun decât pentru ($2, 5$); \par
$\bullet$ Pentru subproblema $(3, 6)$ putem adăuga primul și al treilea obiect aducând un profit mai bun decât profitul anterior pentru $(2, 6)$; \par
$\bullet$ Pentru subproblema $(3, 7)$ putem adăuga al doilea și al treilea obiect obținând un profit de valoare 7 și greutate 7 $ \le j$, mai bun decât profitul anterior pentru $(2, 7)$; \par
$\bullet$ Pentru subproblema $(3, 8)$ se obține un profit mai bun decât profitul calculat pentru subproblema anterioară $(2, 6)$: \par

\resulttable{3}{
    0 & 0 & 0 & 0 & 0 & 0 & 0 & 0 & 0 & 0 \\
    1 & 0 & 0 & 1 & 1 & 1 & 1 & 1 & 1 & 1 \\
    2 & 0 & 0 & 1 & 2 & 2 & 3 & 3 & 3 & 3 \\
    3 & 0 & 0 & 1 & 2 & 5 & 5 & 6 & 7 & 7 \\
}

Asemănător și pentru $i = 4$ când avem toate obiectele la dispoziție: \par
$\bullet$ Pentru subproblema $(4, 5)$ profitul calculat este mai bun decât profitul pentru $(3, 5)$; \par
$\bullet$ Ajungem la $(4, 8)$ care este problema inițială și putem adăuga al doilea și ultimul obiect, iar profitul maxim care se poate obține pornind de la datele de intrare este 8 adunând câștigul obiectului cu rezultatul subproblemei $(3, 3)$:

\resulttable{4}{
    0 & 0 & 0 & 0 & 0 & 0 & 0 & 0 & 0 & 0 \\
    1 & 0 & 0 & 1 & 1 & 1 & 1 & 1 & 1 & 1 \\
    2 & 0 & 0 & 1 & 2 & 2 & 3 & 3 & 3 & 3 \\
    3 & 0 & 0 & 1 & 2 & 5 & 5 & 6 & 7 & 7 \\
    4 & 0 & 0 & 1 & 2 & 5 & 6 & 6 & 7 & 8 \\
}

\par


\end{sloppypar}
