\chapter*{Introducere} 
\addcontentsline{toc}{chapter}{Introducere}

În acest capitol introductiv voi face o scurtă prezentare a limbajului Dafny, urmată apoi de niște informații despre paradigma programării dinamice și câteva detalii despre Problema Rucsacului.

\subsection*{Dafny}
Dafny este un limbaj de programare funcțional destinat verificării formale a corectudinii programelor. A fost creat pentru a ajuta programatorii să scrie cod care este corect din punct de vedere al specificațiilor. Un lucru foarte interesant de știut despre acest limbaj este faptul că a fost conceput astfel încât permite verificarea codului încă din faza de dezvoltare, înainte de a rula codul. Datorită faptului că verificatorul Dafny este rulat ca parte a compilatorului, orice eroare matematică sau logică va fi semnalată către programator care va trebui să ajute verificatorul prin ajustarea specificațiilor sau a logicii. Două exemple foarte simple pentru astfel de specificații sunt următoarele:
\begin{itemize}
    \item requires \( n > 0 \), precondiție care ne va garanta ca numărul de obiecte dat la intrare este mai mare decât 0;
    \item ensures \(sum >  0\), postcondiție care va garanta că suma obținută la ieșire va fi mereu mai mare decât 0.
\end{itemize}

\subsection*{Programarea dinamică}
Programarea dinamică este o metodă de rezolvare a problemelor complexe și se bazează pe conceptul de suprapunere al subproblemelor, însemnând faptul că o problemă mai mare poate fi "spartă" în mai multe subprobleme care sunt mai ușor de rezolvat și care se repetă pe parcursul execuției algoritmului. 
Soluțiile acestor subprobleme sunt stocate astfel încât să nu fie necesară recalcularea lor, lucru care îmbunătățește timpul de rezolvare al  algoritmului. \par 
Principalele abordări prin care soluțiile sunt stocate când folosim programarea dinamică sunt:
\begin{itemize}
     \item \textbf{Memoizarea (Top-Down)} este abordarea recursivă, în care rezultatele sunt salvate într-un tabel de unde vor fi accesate ulterior când este nevoie de rezultatul deja calculat;
     \item \textbf{Tabelizarea (Bottom-Up)} este abordarea în care se pornește de la calcularea celor mai mici subprobleme, construind treptat soluția finală din soluțiile subproblemelor mai mici.
\end{itemize}
În general, paradigma programării dinamice este folosită pentru a rezolva problemele de optimizare care au ca obiectiv obținerea celei mai bune soluții într-un cadru de optimizare, fie pentru minimizare, fie pentru maximizare.\par


\subsection*{Problema Rucsacului}
Problema rucsacului este o problemă de optimizare. Presupunând că avem un rucsac care poate susține o anumită greutate maximă numită capacitate, trebuie sa alegem dintr-o mulțime mai largă de obiecte un număr limitat de obiecte astfel încât valoarea totală a acestora este maximă, iar capacitatea totală a obiectelor nu depășește capacitatea rucsacului. Cele mai cunoscute variațiuni ale acestei probleme sunt varianta continuă și varianta discretă. \par 
În varianta continuă, obiectele pot fi fracționate în mai multe bucăți, permițând alegerea parțială a unui obiect. \par
În varianta discretă fiecare obiect poate fi ales o singură dată sau deloc. Pentru varianta discretă, abordările greedy nu produc mereu o soluție optimă.

\subsection*{Formularea problemei}
Pentru a înțelege mai bine ideea problemei vom considera următorul exemplu:
\begin{textbox}
Date de intrare: 

\end{textbox}