\chapter*{Introducere} 
\addcontentsline{toc}{chapter}{Introducere}

\begin{sloppypar}
În acest capitol introductiv voi face o scurtă prezentare a limbajului Dafny, urmată apoi de niște informații despre paradigma programării dinamice și câteva detalii despre Problema Rucsacului.

\subsection*{Limbajul de programare Dafny}
Dafny este un limbaj de programare funcțional destinat verificării formale a corectudinii programelor. A fost creat pentru a ajuta programatorii să scrie cod care este corect din punct de vedere al specificațiilor. Un lucru foarte interesant de știut despre acest limbaj este faptul că a fost conceput astfel încât permite verificarea codului încă din faza de dezvoltare, înainte de a rula codul. Datorită faptului că verificatorul Dafny este rulat ca parte a compilatorului, orice eroare matematică sau logică va fi semnalată către programator care va trebui să ajute verificatorul prin ajustarea specificațiilor sau a logicii. Două exemple foarte simple pentru astfel de specificații sunt următoarele:
\begin{itemize}
    \item requires \( n > 0 \), precondiție care ne va garanta ca numărul de obiecte dat la intrare este mai mare decât 0;
    \item ensures \(sum >  0\), postcondiție care va garanta că suma obținută la ieșire va fi mereu mai mare decât 0.
\end{itemize}

\subsection*{Programarea dinamică}
Programarea dinamică este o metodă de rezolvare a problemelor complexe și se bazează pe conceptul de suprapunere al subproblemelor, însemnând faptul că o problemă mai mare poate fi "spartă" în mai multe subprobleme care sunt mai ușor de rezolvat și care se repetă pe parcursul execuției algoritmului. 
Soluțiile acestor subprobleme sunt stocate astfel încât să nu fie necesară recalcularea lor, lucru care îmbunătățește timpul de rezolvare al  algoritmului. \par 
Principalele abordări prin care soluțiile sunt stocate când folosim programarea dinamică sunt:
\begin{itemize}
     \item \textbf{Memoizarea (Top-Down)} este abordarea recursivă, în care rezultatele sunt salvate într-un tabel de unde vor fi accesate ulterior când este nevoie de rezultatul deja calculat;
     \item \textbf{Tabelizarea (Bottom-Up)} este abordarea în care se pornește de la calcularea celor mai mici subprobleme, construind treptat soluția finală din soluțiile subproblemelor mai mici.
\end{itemize} \par
În general, paradigma programării dinamice este folosită pentru a rezolva problemele de optimizare care au ca obiectiv obținerea celei mai bune soluții într-un cadru de optimizare, fie pentru minimizare, fie pentru maximizare. \par


\subsection*{Ce este Problema Rucsacului?}
Problema rucsacului este o problemă de optimizare. Presupunând că avem un rucsac care poate susține o anumită greutate maximă numită capacitate, trebuie sa alegem dintr-o mulțime mai largă de obiecte un număr limitat  astfel încât valoarea totală a acestora este maximă, iar greutatea totală a obiectelor nu depășește capacitatea rucsacului. Cele mai cunoscute variațiuni ale acestei probleme sunt varianta continuă și varianta discretă. \par 
În varianta continuă, obiectele pot fi fracționate în mai multe bucăți, permițând alegerea parțială a unui obiect. \par
În varianta discretă fiecare obiect poate fi ales o singură dată sau deloc. Pentru varianta discretă, abordările greedy nu produc mereu o soluție optimă. Astfel vom considera mai departe rezolvarea variantei discrete a problemei rucsacului.

\subsection*{Formularea problemei}
Pentru a înțelege mai bine ideea problemei și cum funcționează algoritmul vom considera următorul exemplu:
\begin{textbox}
\textbf{Date de intrare:} \par
$\bullet$ \textit{n = 4}, unde \textit{n} este numărul total de obiecte pe care le avem la dispoziție; \par
$\bullet$ \textit{c = 8}, unde \textit{c} reprezintă capacitatea totală a rucsacului; \par
$\bullet$ \textit{gains = [1, 2, 5, 6]}, unde \textit{gains} reprezintă un vector de lungime n cu profitul pe care îl aduce fiecare obiect; \par
$\bullet$ \textit{weights = [2, 3, 4, 5]}, unde \textit{weights} reprezintă un vector de lungime n cu greutățile fiecărui obiect.
\end{textbox}

Scopul algoritmului este de a alege obiectele pentru a maximiza profitul obținut respectând în același timp constrângerea de a nu depăși capacitatea rucsacului. Astfel, la fiecare pas algoritmul va trebui sa aleagă una din cele două obțiuni disponibile: fie obiectul va fi adăugat în rucsac integral astfel considerându-se greutatea si profitul adus de acesta, fie obiectul nu va fi adăugat în rucsac. \par
Următoarea schiță a algoritmului implementat poate fi folosită pentru a rezolva varianta discretă a problemei rucsacului:

\begin{tcolorbox}[mystyle]
\begin{enumerate}[left=0.2em]
    \item \textbf{Inițializarea:}
    \begin{itemize}
        \item Inițializăm o matrice \(profits\) de dimensiune \((n + 1) \times (c + 1)\), unde fiecare \( profits[i][j]\) reprezintă profitul maxim posibil obținut pentru o subproblemă în care sunt considerate disponibile maxim \(i\) obiecte, iar rucsacul poate susține o capacitate maximă \(j\);
        \item Inițializăm cazurile de bază:
        \par - prima linie din matrice, care reprezintă cazul în care pentru orice capacitate de la \(0\) până la \(j\) vom considera că nu avem obiecte disponibile. 
        \par - capacitatea rucsacului este \(0\) deoarece profitul maxim posibil care se poate obține este \(0\).
    \end{itemize}
    \item \textbf{Procesul iterativ:}
    Pentru fiecare obiect \(i\) și fiecare capacitate posibilă \(j\) vom considera următoarele cazuri:
    \begin{enumerate}
        \item Obiectul nu poate fi inclus \par
            $\bullet$ greutatea obiectului curent \(weights[i - 1]\) depășește capacitatea \(j\), atunci profitul pentru pasul curent rămâne aceelași ca pentru \(i - 1\) obiecte: \[profits[i][j] = profits[i - 1][j].\]
        \item Obiectul poate fi inclus \par
        Ștind că în acest punct greutatea obiectului nu depășește capacitatea \(j\), avem două opțiuni posibile: \par
        $\bullet$ dacă se obține un profit mai bun decât cel anterior adăugând obiectul, atunci adăugăm valoarea obiectului curent \(gains[i - 1]\) la valoarea obținută pentru capacitatea rămasă \(j - weights[i - 1]\): \[profits[i][j] = profits[i - 1][j = weights[i - 1]] + gains[i - 1].\] \par
        $\bullet$ dacă se obține un profit cel mult la fel de bun ca profitul anterior obținut pentru capacitate \(j\) atunci se va lua profitul maxim obținut fără acest obiect: \[profits[i][j] = profits[i - 1][j].\] 
    \end{enumerate}
    \item \textbf{Rezultatul:}
    La finalul execuției algoritmului, soluția finală care se poate obține se află în \(profits[n][c]\) și reprezintă profitul maxim care se poate obține utilizând toate obiectele pe care le avem la dispoziție pentru un rucsac de capacitate maximă \(c\).
\end{enumerate}
\end{tcolorbox}

Pentru a stoca soluțiile, algoritmul folosește abordarea Bottom-Up în care punctul de start îl reprezintă cazul de bază. După cum am văzut în schița algoritmului prezentat anterior, adeseori pe parcusul execuției vom avea nevoie de profitul calculat în pasul anterior obținut pentru \(i - 1\) obiecte având aceeași capacitate \(j\). Aici intervine proprietatea \textbf{problemelor suprapuse}, în care aceeași subproblemă este rezolvată de mai multe ori în timpul execuției. \par
De asemenea, problema inițială pentru un rucsac de capacitate \(c\) cu \(i\) obiecte poate fi construită astfel din capacitățile \(c - weights[i - 1]\), 
iar fiecare celulă din matricea \textit{profits} stochează rezultatul optim pentru o subproblemă care implică primele \(i\) obiecte și o capacitate \(j\). Am văzut că la fiecare pas al procesului iterativ, mereu se va considera profitul maxim care se poate obține bazat pe două posibile decizii: \par
$\bullet$ Dacă obiectul curent este inclus în soluție, atunci rezultatul va fi calculat din profitul obiectului la care se adaugă soluția subproblemei pentru greutatea rămasă \(j - weights[i - 1]\), soluție care știm că este optimă pentru \(i\) obiecte si capacitate \(j - weights[i - 1]\); \par
$\bullet$ Altfel dacă obiectul nu este inclus, soluția va rămâne aceeași ca cea pentru \(i - 1\) obiecte și capacitate \(j\), soluție care știm că este optimă pentru \(i - 1\) obiecte si capacitate \(j\). \par
Aici intervine cea de-a doua proprietate specifică programării dinamice, mai exact faptul că soluția globală este mereu compusă din soluțiile optime ale subproblemelor, numită proprietatea de \textbf{substructură optimă}.
\\ \par
Aplicând algoritmul pentru a obține profitul maxim pentru datele de intrare de mai sus, voi explica următoarele notații:
\begin{itemize}
    \item $i$ este folosit pentru a memora indexul obiectului
    \item $j$ este folosit pentru a memora indexul capacității parțiale a rucsacului
    \item tuplul $(i = x, j = y)$, unde $0 \le x < n$ și $0 \le j < c$ reprezintă o subproblemă în care considerăm $i$ obiecte și o capacitate parțială $j$ a rucsacului.
\end{itemize}
\par
\textbf{Cazul de bază} îl reprezintă $i = 0$ în care nu avem niciun obiect la dispoziție, iar pentru fiecare subproblemă $(i = 0, j = a)$, unde $0 \le a < c$, soluția optimă este cea de profit maxim 0:

\resulttable{0}{
    0 & 0 & 0 & 0 & 0 & 0 & 0 & 0 & 0 & 0 \\
    }

Algoritmul continuă prin parcurgerea fiecărui obiect și fiecare valoare posibilă a greutății rucsacului. Pentru cazul în care $i = 1$, profitul maxim care se poate obține este 1 deoarece avem un singur obiect la dispoziție: 

\resulttable{1}{
    0 & 0 & 0 & 0 & 0 & 0 & 0 & 0 & 0 & 0 \\
    1 & 0 & 0 & 1 & 1 & 1 & 1 & 1 & 1 & 1 \\
    }

Pentru $i = 2$ considerăm doar primele două obiecte. \par
\begin{itemize}
\item Pentru subproblema $(i = 2, j = 2)$ putem adăuga doar primul obiect în rucsac, iar profitul este 1; 
\item Pentru subproblema $(i = 2, j = 3)$ putem adăuga al doilea obiect deoarece aduce un profit mai bun decât cel anterior $profits[i - 1][j]$ care este 1, iar restricția de a nu depăși capacitatea ruscacului este respectată; 
\item Pentru subproblema $(i = 2, j = 5)$ putem adăuga ambele obiecte în rucsac : 
\end{itemize}

\resulttable{2}{
    0 & 0 & 0 & 0 & 0 & 0 & 0 & 0 & 0 & 0 \\
    1 & 0 & 0 & 1 & 1 & 1 & 1 & 1 & 1 & 1 \\
    2 & 0 & 0 & 1 & 2 & 2 & 3 & 3 & 3 & 3 \\
}

Aplicând aceeași logică și pentru $i = 3$ vom obține următoarele rezultate: \par
$\bullet$ Pentru subproblema $(i = 3, j = 2)$ putem adăuga doar primul obiect; \par
$\bullet$ Pentru subproblema $(i = 3, j = 3)$ putem adăuga doar al doilea obiect; \par
$\bullet$ Pentru subproblema $(i = 3, j = 5)$ putem adăuga al treilea obiect; \par
$\bullet$ Pentru subproblema $(i = 3, j = 6)$ putem adăuga primul și al treilea obiect; \par
$\bullet$ Pentru subproblema $(i = 3, j = 7)$ putem adăuga al doilea și al treilea obiect: \par

\resulttable{3}{
    0 & 0 & 0 & 0 & 0 & 0 & 0 & 0 & 0 & 0 \\
    1 & 0 & 0 & 1 & 1 & 1 & 1 & 1 & 1 & 1 \\
    2 & 0 & 0 & 1 & 2 & 2 & 3 & 3 & 3 & 3 \\
    3 & 0 & 0 & 1 & 2 & 5 & 5 & 6 & 7 & 7 \\
}

Asemănător și pentru $i = 4$ când avem toate obiectele la dispoziție: \par
$\bullet$ Pentru subproblema $(i = 4, j = 2)$ putem adăuga primul obiect; \par
$\bullet$ Pentru subproblema $(i = 4, j = 3)$ putem adăuga doar al doilea obiect; \par
$\bullet$ Pentru subproblema $(i = 4, j = 5)$ putem adăuga al treilea obiect; \par
$\bullet$ Pentru subproblema $(i = 4, j = 6)$ putem adăuga primul și al treilea obiect; \par
$\bullet$ Pentru subproblema $(i = 4, j = 7)$ putem adăuga al doilea și al treilea obiect; \par
$\bullet$ Ajungem la $(i = 4, j = 8)$ care este problema inițială și putem adăuga al doilea și ultimul obiect, iar profitul maxim care se poate obține pornind de la datele de intrare este 8:
\clearpage

\resulttable{4}{
    0 & 0 & 0 & 0 & 0 & 0 & 0 & 0 & 0 & 0 \\
    1 & 0 & 0 & 1 & 1 & 1 & 1 & 1 & 1 & 1 \\
    2 & 0 & 0 & 1 & 2 & 2 & 3 & 3 & 3 & 3 \\
    3 & 0 & 0 & 1 & 2 & 5 & 5 & 6 & 7 & 7 \\
    4 & 0 & 0 & 1 & 2 & 5 & 6 & 6 & 7 & 8 \\
}

\end{sloppypar}
