\chapter*{Motivație} 
\addcontentsline{toc}{chapter}{Motivație}

Am ales să implementez și să demonstrez corectitudinea unui algoritm pentru varianta discretă a problemei rucsacului deoarece întotdeauna am considerat că este o problemă interesantă care de foarte multe ori are aplicabilitate practică și în viața reală. În cazul variantei discrete, abordarea bazată pe programarea dinamică a fost motivată de eficiența pe care această tehnică de rezolvare o oferă în construirea soluțiilor prin identificarea subproblemelor, minimizând astfel redundanța în calcul, dar și garantarea că la finalul algoritmului soluția obținută este mereu optimă. \par
De asemenea, alegerea acestui limbaj a fost motivată de faptul că nu am mai lucrat anterior în Dafny și am considerat că este o oportunitate de a-mi extinde cunoștințele și de a explora noi limbaje cu care nu am interacționat prea mult. De asemenea, am găsit acest limbaj foarte interesant datorită capacității sale de a verifica riguros proprietăți precum corectitudinea și optimalitatea. 