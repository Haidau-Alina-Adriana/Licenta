\lstset{style=mylststyle}
\begin{sloppypar}

\chapter{Reprezentarea datelor de intrare și de ieșire}

În următoarele capitole voi detalia implementarea algoritmului, dar și modul în care Dafny evaluează corectitudinea și optimalitatea codului. \par 
Pentru a reprezenta problema pentru care algoritmul trebuie să obțină cel mai bun profit pe setul de intrare, am ales sa definesc un tip de date $Problem$ astfel:

\begin{Verbatim}[commandchars=\\\{\}]
\PY{k+kd}{datatype} \PY{n+nc}{Problem} \PY{o}{=} \PY{n}{Problem}\PY{p}{(}\PY{n}{n}\PY{p}{:} \PY{k+kt}{int}\PY{p}{,} \PY{n}{c}\PY{p}{:} \PY{k+kt}{int}\PY{p}{,} \PY{n}{gains}\PY{p}{:} \PY{k+kt}{seq}\PY{o}{\PYZlt{}}\PY{k+kt}{int}\PY{o}{\PYZgt{}}\PY{p}{,} 
\PY{n}{weights}\PY{p}{:} \PY{k+kt}{seq}\PY{o}{\PYZlt{}}\PY{k+kt}{int}\PY{o}{\PYZgt{}}\PY{p}{)} 
\end{Verbatim}
unde:
\begin{itemize}
    \item câmpul \textit{\textbf{\textcolor{coleight}{n}}} reprezintă numărul total de obiecte disponibile;
    \item câmpul \textit{\textbf{\textcolor{coleight}{c}}} reprezintă capacitatea totală a rucsacului;
    \item câmpul \textit{\textbf{\textcolor{coleight}{gains}}} este o secvență ce memorează profitul fiecărui obiect;
    \item câmpul \textit{\textbf{\textcolor{coleight}{weights}}} este o secvență ce memorează greutatea fiecărui obiect.
\end{itemize} \par
Folosind cuvântul cheie \formatText{datatype} am definit un nou tip de date inductiv numit $Problem$ cu un singur constructor având același nume în care câmpurile sunt separate prin virgulă.
\par
Atât profitul, cât și greutatea fiecărui obiect pot fi accesate prin indexul corespunzător poziției din secvență. Spre exemplu, pentru a accesa greutatea celui de-al doilea obiect este folosit operatorul de indexare $weights[1]$, deoarece indexarea celor două secvențe începe de la 0. În dafny, o \textbf{secvență} este o colecție de elemente de același tip, iar în cazul acesta, de numere întregi.  \par
Orice instanța a problemei oferită la intrare trebuie să fie validă și să respecte anumite condiții pentru ca algoritmul să poată produce o soluție corectă. Pentru a mă asigura că aceste condiții sunt îndeplinite a fost nevoie să formulez un predicat. \textbf{Predicatele} în Dafny sunt funcții ale călor rezultate sunt valori boolene. Acestea sunt folosite pentru a evalua proprietăți care de obicei trebuie să fie adevărate și de aceea ele ajută în procesul de verificare al corectitudinii. Astfel, am definit predicatul \formatText{isValidProblem} care verifică dacă avem cel puțin un obiect la dispoziție și un rucsac de capacitate mai mare decât zero, dacă știm greutatea și câștigul fiecărui obiect, iar acestea au valori pozitive diferite de zero:
    \begin{Verbatim}[commandchars=\\\{\}]
\PY{k+kd}{predicate} \PY{n+nf}{isValidProblem}\PY{p}{(}\PY{n}{p}\PY{p}{:} \PY{n}{Problem}\PY{p}{)}
\PY{p}{\PYZob{}}
  \PY{o}{|}\PY{n}{p}\PY{p}{.}\PY{n}{gains}\PY{o}{|} \PY{o}{==} \PY{o}{|}\PY{n}{p}\PY{p}{.}\PY{n}{weights}\PY{o}{|} \PY{o}{==} \PY{n}{p}\PY{p}{.}\PY{n}{n} \PY{o}{\PYZam{}\PYZam{}} 
  \PY{n}{p}\PY{p}{.}\PY{n}{n} \PY{o}{\PYZgt{}} \PY{l+m+mi}{0} \PY{o}{\PYZam{}\PYZam{}} \PY{n}{p}\PY{p}{.}\PY{n}{c} \PY{o}{\PYZgt{}=} \PY{l+m+mi}{0} \PY{o}{\PYZam{}\PYZam{}} 
  \PY{n}{hasPositiveValues}\PY{p}{(}\PY{n}{p}\PY{p}{.}\PY{n}{gains}\PY{p}{)} \PY{o}{\PYZam{}\PYZam{}} \PY{n}{hasPositiveValues}\PY{p}{(}\PY{n}{p}\PY{p}{.}\PY{n}{weights}\PY{p}{)} 
\PY{p}{\PYZcb{}}
\end{Verbatim}
\par Predicatul \formatText{hasPositiveValues} este definit astfel:
    \begin{Verbatim}[commandchars=\\\{\}]
\PY{k+kd}{predicate} \PY{n+nf}{hasPositiveValues}\PY{p}{(}\PY{n}{arr}\PY{p}{:} \PY{k+kt}{seq}\PY{o}{\PYZlt{}}\PY{k+kt}{int}\PY{o}{\PYZgt{}}\PY{p}{)}
\PY{p}{\PYZob{}}
  \PY{k}{forall} \PY{n}{i} \PY{p}{::} \PY{l+m+mi}{0} \PY{o}{\PYZlt{}=} \PY{n}{i} \PY{o}{\PYZlt{}} \PY{o}{|}\PY{n}{arr}\PY{o}{|} \PY{o}{==}\PY{o}{\PYZgt{}} \PY{n}{arr}\PY{p}{[}\PY{n}{i}\PY{p}{]} \PY{o}{\PYZgt{}} \PY{l+m+mi}{0}
\PY{p}{\PYZcb{}}
\end{Verbatim}
    și este folosit, după cum sugerează și numele acestuia, pentru a verifica dacă toate elementele unei secvențe precum $gains$ sunt pozitive. \\ \par 

Datele de ieșire sunt reprezentate astfel:
\begin{Verbatim}[commandchars=\\\{\}]
                \PY{p}{(}\PY{n}{profit}\PY{p}{:} \PY{k+kt}{int}\PY{p}{,} \PY{n}{solution}\PY{p}{:} \PY{n}{Solution}\PY{p}{)}
\end{Verbatim}
unde:
\begin{itemize}
    \item \formatText{profit} reprezintă profitul maxim care se poate obține pentru o anumită instanță a problemei;
    \item \formatText{solution} reprezintă un sinonim al tipului de date \texttt{seq\(<\)int\(>\)} folosit pentru a îmbunătăți claritatea codului, este declarat anterior astfel:
\begin{Verbatim}[commandchars=\\\{\}]
                \PY{n}{type} \PY{n}{Solution} \PY{o}{=} \PY{k+kt}{seq}\PY{o}{\PYZlt{}}\PY{k+kt}{int}\PY{o}{\PYZgt{}}
\end{Verbatim}
și este o reprezentare binară a deciziei de includere a obiectului în rucsac. Prin urmare, valoarea de 1 reprezintă faptul că obiectul aparține soluției, deci este adăugat în rucsac, iar profitul adus de acesta îmbunătățește câștigul final, iar 0 reprezintă faptul că obiectul nu este adăugat în rucsac, deci profitul obiectului nu este inclus în câștigul maxim obținut. Această secvență reprezintă soluția finală care trebuie să fie optimă pentru problema completă.
\end{itemize} 

\end{sloppypar}