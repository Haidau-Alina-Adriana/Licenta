\lstset{style=mylststyle}

\chapter{Detalii de implementare ale problemei}
\begin{sloppypar}
În acest capitol voi detalia implementarea algoritmului, dar și modul în care Dafny evaluează corectitudinea și optimalitatea codului. 

\section{Reprezentarea datelor de intrare/ieșire}

Pentru a reprezenta problema pentru care algoritmul trebuie să obțină cel mai bun profit pe setul de intrare, am ales sa definesc un tip de date \textbf{\textit{\textcolor{coleight}{Problem}}} astfel:

\begin{Verbatim}[commandchars=\\\{\}]
\PY{k+kd}{datatype} \PY{n+nc}{Problem} \PY{o}{=} \PY{n}{Problem}\PY{p}{(}\PY{n}{n}\PY{p}{:} \PY{k+kt}{int}\PY{p}{,} \PY{n}{c}\PY{p}{:} \PY{k+kt}{int}\PY{p}{,} \PY{n}{gains}\PY{p}{:} \PY{k+kt}{seq}\PY{o}{\PYZlt{}}\PY{k+kt}{int}\PY{o}{\PYZgt{}}\PY{p}{,} 
\PY{n}{weights}\PY{p}{:} \PY{k+kt}{seq}\PY{o}{\PYZlt{}}\PY{k+kt}{int}\PY{o}{\PYZgt{}}\PY{p}{)} 
\end{Verbatim}
unde:
\begin{itemize}
    \item câmpul \textit{\textbf{\textcolor{coleight}{n}}} reprezintă numărul de obiecte disponibile;
    \item câmpul \textit{\textbf{\textcolor{coleight}{c}}} reprezintă capacitatea totală a rucsacului;
    \item câmpul \textit{\textbf{\textcolor{coleight}{gains}}} este o secvență ce memorează profitul fiecărui obiect;
    \item câmpul \textit{\textbf{\textcolor{coleight}{weights}}} este o secvență ce memorează greutatea fiecărui obiect.
\end{itemize} \par
Folosind cuvântul cheie \textit{\textbf{\textcolor{coleight}{datatype}}} am definit un nou tip de date inductiv numit \textit{\textbf{\textcolor{coleight}{Problem}}} cu un singur constructor având același nume în care câmpurile sunt separate prin virgulă și sunt tipuri de date primitive.
\par
Atât profitul, cât și greutatea fiecărui obiect pot fi accesate prin indexul corespunzător poziției din secvență. Spre exemplu, pentru a accesa greutatea celui de-al doilea obiect este folosit operatorul de indexare $weights[1]$, deoarece indexarea celor două secvențe începe de la 0. În dafny, o \textit{secvență} este o colecție de elemente de același tip, iar în cazul acesta, de numere întregi. \\ \par

Datele de ieșire sunt reprezentate astfel:
\begin{Verbatim}[commandchars=\\\{\}]
                \PY{p}{(}\PY{n}{profit}\PY{p}{:} \PY{k+kt}{int}\PY{p}{,} \PY{n}{solution}\PY{p}{:} \PY{n}{Solution}\PY{p}{)}
\end{Verbatim}
unde:
\begin{itemize}
    \item \textit{\textbf{\textcolor{coleight}{profit}}} reprezintă profitul maxim care se poate obține pentru o anumită instanță a problemei;
    \item \textit{\textbf{\textcolor{coleight}{solution}}} reprezintă un sinonim al tipului de date \texttt{seq\(<\)int\(>\)} folosit pentru a îmbunătăți claritatea codului, este declarat anterior astfel:
\begin{Verbatim}[commandchars=\\\{\}]
                \PY{n}{type} \PY{n}{Solution} \PY{o}{=} \PY{k+kt}{seq}\PY{o}{\PYZlt{}}\PY{k+kt}{int}\PY{o}{\PYZgt{}}
\end{Verbatim}
și este o reprezentare binară a deciziei de includere a obiectului în rucsac. Prin urmare, valoarea de 1 reprezintă faptul că obiectul aparține soluției, deci este adăugat în rucsac, iar profitul adus de acesta îmbunătățește câștigul final, iar 0 reprezintă faptul că obiectul nu este adăugat în rucsac, deci profitul obiectului nu este inclus în câștigul maxim obținut. Această secvență reprezintă soluția finală și trebuie să fie optimă.
\end{itemize} 

\end{sloppypar}

\section{Soluția optimă și soluțiile parțial optime}

\begin{sloppypar}

În cadrul problemei rucsacului putem discuta despre ce reprezintă soluția optimă, cât și soluțiile parțial optime. \par
Pentru a stoca rezultatele corespunzătoare câștigurilor subproblemelor am folosit o matrice numită $profits$ de tip \(seq<seq<int>>\), iar rezultatele pot fi accesate folosind operatorul de indexare pentru un tablou bidimensional: $profits[i][j]$, unde $i$ și $j$ reprezită mărimea subproblemei pe care încercăm să o rezolvăm.
\par
Pe lângă matricea $profits$, această lucrare se bazează pe o matrice suplimentară numită $solutions$ de tip \(seq<seq<seq<int>>>\) care va stoca fiecare secvență binară corespunzătoare profitului stocat în matricea $profits$. Rezultatele sunt accesate similar matricei $profits$.
\par
\textbf{O soluție optimă} reprezintă selecția de obiecte care maximizează profitul format din câștigul fiecărui item care poate fi adăugat în rucsac, dar care în același timp respectă constrângerile legate de capacitatea rucsacului pentru problema completă. Acest rezultat trebuie să fie cel mai bun pe care îl putem obține pentru instanța problemei pe care o primim la intrare, iar această soluție este găsită în ultima celulă a matricei, $solutions[n][c]$.
\par
\textbf{O soluție parțială} este o secvență binară de o lungime variabilă (care nu depășește numărul de obiecte) corespunzătoare deciziei de a include un obiect care respectă constrângerea legată de capacitatea rucsacului. \par
\textbf{O soluție parțial optimă} reprezintă un rezultat optim intermediar calculat pentru un subset de obiecte sau pentru o capacitate parțială a rucsacului. Luând exemplul prezentat în introducere, o soluție parțială pentru subproblema $(i = 3, j = 8)$ este $[0, 1, 1]$ și aduce cel mai bun profit pentru un subset ce include doar primele trei obiecte, având la dispoziție un rucsac de capacitate maximă 8.

\end{sloppypar}

\section{Precondiții, postcondiții și invarianți}

Precondițiile, postcondițiile și invarianții mai sunt numite și specificații și reprezintă proprietăți logice folosite pentru a descrie comportamentul pe care ar trebui să îl îndeplinească programul la un anumit punct. Ele garantează că aceste proprietăți sunt respectate pe tot parcursul execuției și sunt folosite pentru metode, funcții, bucle, leme și iteratori. \par
Specificațiile sunt esențiale pentru ajutarea verificatorului, permițând astfel validarea corectitudinii logice a programului. Vom considera mai departe această bucată de cod ce aparține unei leme folosite pentru demonstrarea faptului că o soluție este parțială prin adăugarea unui obiect:
\begin{Verbatim}[commandchars=\\\{\}]
\PY{k+kd}{lemma} \PY{n+nf}{computeWeightFits1}\PY{p}{(}\PY{n}{p}\PY{p}{:} \PY{n}{Problem}\PY{p}{,} \PY{n}{solution}\PY{p}{:} \PY{n}{Solution}\PY{p}{,} 
    \PY{n}{i}\PY{p}{:} \PY{k+kt}{int}\PY{p}{,} \PY{n}{j}\PY{p}{:} \PY{k+kt}{int}\PY{p}{)}
  \PY{k}{requires} \PY{n}{isValidProblem}\PY{p}{(}\PY{n}{p}\PY{p}{)}
  \PY{k}{requires} \PY{l+m+mi}{0} \PY{o}{\PYZlt{}=} \PY{o}{|}\PY{n}{solution}\PY{o}{|} \PY{o}{\PYZlt{}} \PY{n}{p}\PY{p}{.}\PY{n}{n}
  \PY{k}{requires} \PY{n}{hasAllowedValues}\PY{p}{(}\PY{n}{solution}\PY{p}{)}
  \PY{k}{requires} \PY{l+m+mi}{0} \PY{o}{\PYZlt{}=} \PY{n}{j} \PY{o}{\PYZlt{}=} \PY{n}{p}\PY{p}{.}\PY{n}{c} \PY{o}{+} \PY{l+m+mi}{1}
  \PY{k}{requires} \PY{n}{i} \PY{o}{==} \PY{o}{|}\PY{n}{solution}\PY{o}{|}
  \PY{k}{requires} \PY{n}{weight}\PY{p}{(}\PY{n}{p}\PY{p}{,} \PY{n}{solution}\PY{p}{)} \PY{o}{\PYZlt{}=} \PY{n}{j} \PY{o}{\PYZhy{}} \PY{n}{p}\PY{p}{.}\PY{n}{weights}\PY{p}{[}\PY{n}{i}\PY{p}{]}
  
  \PY{k}{ensures} \PY{n}{computeWeight}\PY{p}{(}\PY{n}{p}\PY{p}{,} \PY{n}{solution} \PY{o}{+} \PY{p}{[}\PY{l+m+mi}{1}\PY{p}{]}\PY{p}{,} \PY{o}{|}\PY{n}{solution} \PY{o}{+} \PY{p}{[}\PY{l+m+mi}{1}\PY{p}{]}\PY{o}{|} \PY{o}{\PYZhy{}} \PY{l+m+mi}{1}\PY{p}{)} \PY{o}{\PYZlt{}=} \PY{n}{j}
\PY{p}{\PYZob{}}
  \PY{k+kd}{var} \PY{n}{s} \PY{o}{:=} \PY{n}{solution} \PY{o}{+} \PY{p}{[}\PY{l+m+mi}{1}\PY{p}{]}\PY{p}{;}
  \PY{k}{assert} \PY{n}{solution} \PY{o}{==} \PY{n}{s}\PY{p}{[}\PY{p}{..}\PY{o}{|}\PY{n}{s}\PY{o}{|} \PY{o}{\PYZhy{}} \PY{l+m+mi}{1}\PY{p}{]}\PY{p}{;}

  \PY{n}{for} \PY{n}{a} \PY{o}{:=} \PY{l+m+mi}{0} \PY{n}{to} \PY{o}{|}\PY{n}{s}\PY{p}{[}\PY{p}{..}\PY{o}{|}\PY{n}{s}\PY{o}{|} \PY{o}{\PYZhy{}} \PY{l+m+mi}{1}\PY{p}{]}\PY{o}{|}
    \PY{k}{invariant} \PY{l+m+mi}{0} \PY{o}{\PYZlt{}=} \PY{n}{a} \PY{o}{\PYZlt{}=} \PY{o}{|}\PY{n}{s}\PY{p}{[}\PY{p}{..}\PY{o}{|}\PY{n}{s}\PY{o}{|} \PY{o}{\PYZhy{}} \PY{l+m+mi}{1}\PY{p}{]}\PY{o}{|} \PY{o}{+} \PY{l+m+mi}{1}
    \PY{k}{invariant} \PY{k}{forall} \PY{n}{k} \PY{p}{::} \PY{l+m+mi}{0} \PY{o}{\PYZlt{}=} \PY{n}{k} \PY{o}{\PYZlt{}} \PY{n}{a} \PY{o}{==}\PY{o}{\PYZgt{}} 
        \PY{n}{computeWeight}\PY{p}{(}\PY{n}{p}\PY{p}{,} \PY{n}{solution}\PY{p}{,} \PY{n}{k}\PY{p}{)} \PY{o}{==} \PY{n}{computeWeight}\PY{p}{(}\PY{n}{p}\PY{p}{,} \PY{n}{s}\PY{p}{,} \PY{n}{k}\PY{p}{)}
  \PY{p}{\PYZob{}} \PY{p}{\PYZcb{}}
\PY{p}{\PYZcb{}}
\end{Verbatim}
Astfel avem:
\begin{itemize}
    \item \textit{\textbf{\textcolor{coleight}{Precondițiile}}} sunt expresii logice care trebuie să fie adevărate înainte de executarea logicii unei functii sau metode și necesită clauza \texttt{requires}:
    \begin{Verbatim}[commandchars=\\\{\}]
  \PY{k}{requires} \PY{n}{isValidProblem}\PY{p}{(}\PY{n}{p}\PY{p}{)}
\end{Verbatim}
Această precondiție, spre exemplu, ne asigură că vom lucra în cadrul lemei cu o problemă validă.
\item \textit{\textbf{\textcolor{coleight}{Postcondițiile}}} sunt expresii logice care trebuie să fie adevărate după executarea logicii unei funcții sau metode și necesită clauza \texttt{ensures}:
    \begin{Verbatim}[commandchars=\\\{\}]
  \PY{k}{ensures} \PY{n}{computeWeight}\PY{p}{(}\PY{n}{p}\PY{p}{,} \PY{n}{solution} \PY{o}{+} \PY{p}{[}\PY{l+m+mi}{1}\PY{p}{]}\PY{p}{,} \PY{o}{|}
    \PY{n}{solution} \PY{o}{+} \PY{p}{[}\PY{l+m+mi}{1}\PY{p}{]}\PY{o}{|} \PY{o}{\PYZhy{}} \PY{l+m+mi}{1}\PY{p}{)} \PY{o}{\PYZlt{}=} \PY{n}{j}
\end{Verbatim}
    Singura postcondiție ne asigură faptul că la ieșirea din lemă putem garanta că pentru parametrii de intrare care respectă precondițiile, expresia logică din postcondiție este adevarată.
    \item \textit{\textbf{\textcolor{coleight}{Invarianții}}} sunt expresii logice care de obieci sunt folosiți pentru specificațiile unei bucle și trebuie să fie adevărate pe toată durata ciclului, inclusiv înainte de intrarea în buclă, cât și după ieșirea din aceasta. Pentru invarianți se folosește clauza \texttt{invariant}:
    \begin{Verbatim}[commandchars=\\\{\}]
\PY{k}{invariant} \PY{l+m+mi}{0} \PY{o}{\PYZlt{}=} \PY{n}{a} \PY{o}{\PYZlt{}=} \PY{o}{|}\PY{n}{s}\PY{p}{[}\PY{p}{..}\PY{o}{|}\PY{n}{s}\PY{o}{|} \PY{o}{\PYZhy{}} \PY{l+m+mi}{1}\PY{p}{]}\PY{o}{|} \PY{o}{+} \PY{l+m+mi}{1}
\end{Verbatim}
    Invariantul acesta verifică respectarea limitelor valorilor luate de \texttt{a} în cadrul buclei \texttt{for}. Acest invariant este verificat înainte de începerea executării buclei \texttt{for}, cât și pe parcursul fiecărei iterații și după ieșirea din buclă.
\end{itemize}

\section{Predicate și Funcții}

\subsection{Predicate}

Predicatele în Dafny sunt funcții ale călor rezultate sunt valori boolene. Acestea sunt folosite pentru a evalua proprietăți care de obicei trebuie să fie adevărate și de aceea ele ajută în procesul de verificare al corectitudinii. Predicatele pe care urmează să le prezint au fost foarte des folosite ca și condiții pe care metodele și lemele trebuie să le respecte, dar și ca rezultate la ieșirea din acestea pentru asigurarea unor proprietăți necesare, de exemplu, în verificarea optimalității. \par
Astfel, voi continua cu prezentarea acestor predicate:
\begin{enumerate}
    \item Predicatul \textit{hasPositiveValues} este definit astfel:
    \begin{Verbatim}[commandchars=\\\{\}]
\PY{k+kd}{predicate} \PY{n+nf}{hasPositiveValues}\PY{p}{(}\PY{n}{arr}\PY{p}{:} \PY{k+kt}{seq}\PY{o}{\PYZlt{}}\PY{k+kt}{int}\PY{o}{\PYZgt{}}\PY{p}{)}
\PY{p}{\PYZob{}}
  \PY{k}{forall} \PY{n}{i} \PY{p}{::} \PY{l+m+mi}{0} \PY{o}{\PYZlt{}=} \PY{n}{i} \PY{o}{\PYZlt{}} \PY{o}{|}\PY{n}{arr}\PY{o}{|} \PY{o}{==}\PY{o}{\PYZgt{}} \PY{n}{arr}\PY{p}{[}\PY{n}{i}\PY{p}{]} \PY{o}{\PYZgt{}} \PY{l+m+mi}{0}
\PY{p}{\PYZcb{}}
\end{Verbatim}
    și este folosit, după cum sugerează și numele acestuia, pentru a verifica dacă toate elementele unui vector precum $gains$ sunt pozitive.
    \item Predicatul \textit{hasAllowedValues}
    \begin{Verbatim}[commandchars=\\\{\}]
\PY{k+kd}{predicate} \PY{n+nf}{hasAllowedValues}\PY{p}{(}\PY{n}{solution}\PY{p}{:} \PY{n}{Solution}\PY{p}{)}
\PY{p}{\PYZob{}}
  \PY{k}{forall} \PY{n}{k} \PY{p}{::} \PY{l+m+mi}{0} \PY{o}{\PYZlt{}=} \PY{n}{k} \PY{o}{\PYZlt{}} \PY{o}{|}\PY{n}{solution}\PY{o}{|} \PY{o}{==}\PY{o}{\PYZgt{}} \PY{n}{solution}\PY{p}{[}\PY{n}{k}\PY{p}{]} \PY{o}{==} \PY{l+m+mi}{0} 
        \PY{o}{|}\PY{o}{|} \PY{n}{solution}\PY{p}{[}\PY{n}{k}\PY{p}{]} \PY{o}{==} \PY{l+m+mi}{1}
\PY{p}{\PYZcb{}}
\end{Verbatim}
    se asigură că toate elementele unei soluții aparțin exclusiv mulțimii $\{0, 1\}$, unde 0 \(-\) obiectul nu este în rucsac, 1 \(-\) obiectul este în rucsac.
    \item Predicatul \textit{isValidProblem}, definit astfel:
    \begin{Verbatim}[commandchars=\\\{\}]
\PY{k+kd}{predicate} \PY{n+nf}{isValidProblem}\PY{p}{(}\PY{n}{p}\PY{p}{:} \PY{n}{Problem}\PY{p}{)}
\PY{p}{\PYZob{}}
  \PY{o}{|}\PY{n}{p}\PY{p}{.}\PY{n}{gains}\PY{o}{|} \PY{o}{==} \PY{o}{|}\PY{n}{p}\PY{p}{.}\PY{n}{weights}\PY{o}{|} \PY{o}{==} \PY{n}{p}\PY{p}{.}\PY{n}{n} \PY{o}{\PYZam{}\PYZam{}} 
  \PY{n}{p}\PY{p}{.}\PY{n}{n} \PY{o}{\PYZgt{}} \PY{l+m+mi}{0} \PY{o}{\PYZam{}\PYZam{}} \PY{n}{p}\PY{p}{.}\PY{n}{c} \PY{o}{\PYZgt{}=} \PY{l+m+mi}{0} \PY{o}{\PYZam{}\PYZam{}} 
  \PY{n}{hasPositiveValues}\PY{p}{(}\PY{n}{p}\PY{p}{.}\PY{n}{gains}\PY{p}{)} \PY{o}{\PYZam{}\PYZam{}} \PY{n}{hasPositiveValues}\PY{p}{(}\PY{n}{p}\PY{p}{.}\PY{n}{weights}\PY{p}{)} 
\PY{p}{\PYZcb{}}
\end{Verbatim}
    este un predicat foarte important, folosit ca precondiție aproape pentru fiecare funcție și metodă din cod. Acest predicat este folosit ca validator pentru instanța problemei a cărei soluție optimă încercăm să găsim. Ne așteptăm astfel să avem cel puțin un obiect la dispoziție și un rucsac de capacitate mai mare decât zero, să știm greutatea și câștigul fiecărui obiect, iar acestea să fie la rândul lor valori pozitive diferite de zero.
    \item Predicatul \textit{isValidPartialSolution}
    \begin{Verbatim}[commandchars=\\\{\}]
\PY{k+kd}{predicate} \PY{n+nf}{isValidPartialSolution}\PY{p}{(}\PY{n}{p}\PY{p}{:} \PY{n}{Problem}\PY{p}{,} \PY{n}{solution}\PY{p}{:} \PY{n}{Solution}\PY{p}{)}
  \PY{k}{requires} \PY{n}{isValidProblem}\PY{p}{(}\PY{n}{p}\PY{p}{)}
\PY{p}{\PYZob{}}
  \PY{n}{hasAllowedValues}\PY{p}{(}\PY{n}{solution}\PY{p}{)} \PY{o}{\PYZam{}\PYZam{}} \PY{o}{|}\PY{n}{solution}\PY{o}{|} \PY{o}{\PYZlt{}=} \PY{n}{p}\PY{p}{.}\PY{n}{n}
\PY{p}{\PYZcb{}}
\end{Verbatim}
    este folosit pentru a verifica dacă soluția parțială este validă, mai exact dacă are doar elemente de 0 și 1, iar lungimea acesteia este cel mult egală cu numărul de obiecte ale instanței.
    \item Predicatul \textit{isPartialSolution}
    \begin{Verbatim}[commandchars=\\\{\}]
\PY{k+kd}{predicate} \PY{n+nf}{isPartialSolution}\PY{p}{(}\PY{n}{p}\PY{p}{:} \PY{n}{Problem}\PY{p}{,} \PY{n}{solution}\PY{p}{:} \PY{n}{Solution}\PY{p}{,} 
        \PY{n}{i}\PY{p}{:} \PY{k+kt}{int}\PY{p}{,} \PY{n}{j}\PY{p}{:} \PY{k+kt}{int}\PY{p}{)}
  \PY{k}{requires} \PY{n}{isValidProblem}\PY{p}{(}\PY{n}{p}\PY{p}{)}
  \PY{k}{requires} \PY{l+m+mi}{0} \PY{o}{\PYZlt{}=} \PY{n}{i} \PY{o}{\PYZlt{}=} \PY{n}{p}\PY{p}{.}\PY{n}{n}
  \PY{k}{requires} \PY{l+m+mi}{0} \PY{o}{\PYZlt{}=} \PY{n}{j} \PY{o}{\PYZlt{}=} \PY{n}{p}\PY{p}{.}\PY{n}{c}
\PY{p}{\PYZob{}}
  \PY{n}{isValidPartialSolution}\PY{p}{(}\PY{n}{p}\PY{p}{,} \PY{n}{solution}\PY{p}{)} \PY{o}{\PYZam{}\PYZam{}} \PY{o}{|}\PY{n}{solution}\PY{o}{|} \PY{o}{==} \PY{n}{i} \PY{o}{\PYZam{}\PYZam{}}
  \PY{n}{weight}\PY{p}{(}\PY{n}{p}\PY{p}{,} \PY{n}{solution}\PY{p}{)} \PY{o}{\PYZlt{}=} \PY{n}{j}
\PY{p}{\PYZcb{}}
\end{Verbatim}
    vine în completarea predicatului \textit{isValidPartialSolution} și pe lângă proprietățile aduse de acesta, verifică dacă soluția este validă pentru subproblema $(i, j)$.
    \item Predicatul \textit{isSolution}
    \begin{Verbatim}[commandchars=\\\{\}]
\PY{k+kd}{predicate} \PY{n+nf}{isSolution}\PY{p}{(}\PY{n}{p}\PY{p}{:} \PY{n}{Problem}\PY{p}{,} \PY{n}{solution}\PY{p}{:} \PY{n}{Solution}\PY{p}{)}
  \PY{k}{requires} \PY{n}{isValidProblem}\PY{p}{(}\PY{n}{p}\PY{p}{)}
\PY{p}{\PYZob{}}
  \PY{n}{isValidPartialSolution}\PY{p}{(}\PY{n}{p}\PY{p}{,} \PY{n}{solution}\PY{p}{)} \PY{o}{\PYZam{}\PYZam{}} \PY{o}{|}\PY{n}{solution}\PY{o}{|} \PY{o}{==} \PY{n}{p}\PY{p}{.}\PY{n}{n} \PY{o}{\PYZam{}\PYZam{}}
  \PY{n}{weight}\PY{p}{(}\PY{n}{p}\PY{p}{,} \PY{n}{solution}\PY{p}{)} \PY{o}{\PYZlt{}=} \PY{n}{p}\PY{p}{.}\PY{n}{c}
\PY{p}{\PYZcb{}}
\end{Verbatim} 
    verifică dacă avem o soluție pentru problema completă.
    \item Predicatul \textit{isOptimalPartialSolution}
    \begin{Verbatim}[commandchars=\\\{\}]
\PY{k+kd}{ghost} \PY{k+kd}{predicate} \PY{n+nf}{isOptimalPartialSolution}\PY{p}{(}\PY{n}{p}\PY{p}{:} \PY{n}{Problem}\PY{p}{,} 
    \PY{n}{solution}\PY{p}{:} \PY{n}{Solution}\PY{p}{,} \PY{n}{i}\PY{p}{:} \PY{k+kt}{int}\PY{p}{,} \PY{n}{j}\PY{p}{:} \PY{k+kt}{int}\PY{p}{)}
  \PY{k}{requires} \PY{n}{isValidProblem}\PY{p}{(}\PY{n}{p}\PY{p}{)}
  \PY{k}{requires} \PY{l+m+mi}{0} \PY{o}{\PYZlt{}=} \PY{n}{i} \PY{o}{\PYZlt{}=} \PY{n}{p}\PY{p}{.}\PY{n}{n}
  \PY{k}{requires} \PY{l+m+mi}{0} \PY{o}{\PYZlt{}=} \PY{n}{j} \PY{o}{\PYZlt{}=} \PY{n}{p}\PY{p}{.}\PY{n}{c}
\PY{p}{\PYZob{}}
  \PY{n}{isPartialSolution}\PY{p}{(}\PY{n}{p}\PY{p}{,} \PY{n}{solution}\PY{p}{,} \PY{n}{i}\PY{p}{,} \PY{n}{j}\PY{p}{)} \PY{o}{\PYZam{}\PYZam{}}
  \PY{k}{forall} \PY{n}{s}\PY{p}{:} \PY{n}{Solution} \PY{p}{::} \PY{p}{(}\PY{n}{isPartialSolution}\PY{p}{(}\PY{n}{p}\PY{p}{,} \PY{n}{s}\PY{p}{,} \PY{n}{i}\PY{p}{,} \PY{n}{j}\PY{p}{)} \PY{o}{\PYZam{}\PYZam{}} 
    \PY{o}{|}\PY{n}{s}\PY{o}{|} \PY{o}{==} \PY{o}{|}\PY{n}{solution}\PY{o}{|} \PY{o}{==}\PY{o}{\PYZgt{}} \PY{n}{gain}\PY{p}{(}\PY{n}{p}\PY{p}{,} \PY{n}{solution}\PY{p}{)} \PY{o}{\PYZgt{}=} \PY{n}{gain}\PY{p}{(}\PY{n}{p}\PY{p}{,} \PY{n}{s}\PY{p}{)}\PY{p}{)}
\PY{p}{\PYZcb{}}
\end{Verbatim}
    este folosit pentru verificarea optimalității unei subprobleme $(i, j)$, adică avem cel mai bun câștig care se poate obține pentru orice soluție validă cu elemente de 0 și 1 ale unei subprobleme cu $i$ obiecte și o capacitate $j$ a rucsacului.
    \item Predicatul \textit{isOptimalSolution}
\begin{Verbatim}[commandchars=\\\{\}]
\PY{k+kd}{ghost} \PY{k+kd}{predicate} \PY{n+nf}{isOptimalSolution}\PY{p}{(}\PY{n}{p}\PY{p}{:} \PY{n}{Problem}\PY{p}{,} 
    \PY{n}{solution}\PY{p}{:} \PY{n}{Solution}\PY{p}{)}
  \PY{k}{requires} \PY{n}{isValidProblem}\PY{p}{(}\PY{n}{p}\PY{p}{)}
  \PY{k}{requires} \PY{n}{isValidPartialSolution}\PY{p}{(}\PY{n}{p}\PY{p}{,} \PY{n}{solution}\PY{p}{)}
\PY{p}{\PYZob{}}
  \PY{n}{isOptimalPartialSolution}\PY{p}{(}\PY{n}{p}\PY{p}{,} \PY{n}{solution}\PY{p}{,} \PY{n}{p}\PY{p}{.}\PY{n}{n}\PY{p}{,} \PY{n}{p}\PY{p}{.}\PY{n}{c}\PY{p}{)} \PY{o}{\PYZam{}\PYZam{}}
  \PY{k}{forall} \PY{n}{s}\PY{p}{:} \PY{n}{Solution} \PY{p}{::} \PY{p}{(}\PY{p}{(}\PY{p}{(}\PY{n}{isOptimalPartialSolution}\PY{p}{(}
    \PY{n}{p}\PY{p}{,} \PY{n}{s}\PY{p}{,} \PY{n}{p}\PY{p}{.}\PY{n}{n}\PY{p}{,} \PY{n}{p}\PY{p}{.}\PY{n}{c}\PY{p}{)}\PY{p}{)} \PY{o}{==}\PY{o}{\PYZgt{}} \PY{n}{gain}\PY{p}{(}\PY{n}{p}\PY{p}{,} \PY{n}{solution}\PY{p}{)} \PY{o}{\PYZgt{}=} \PY{n}{gain}\PY{p}{(}\PY{n}{p}\PY{p}{,} \PY{n}{s}\PY{p}{)}\PY{p}{)}\PY{p}{)}
\PY{p}{\PYZcb{}}
\end{Verbatim}
    verifică optimalitatea soluției finale.
    \item Predicatul \textit{isValidSubproblem}
    \begin{Verbatim}[commandchars=\\\{\}]
\PY{k+kd}{predicate} \PY{n+nf}{isValidSubproblem}\PY{p}{(}\PY{n}{p}\PY{p}{:} \PY{n}{Problem}\PY{p}{,} \PY{n}{i}\PY{p}{:} \PY{k+kt}{int}\PY{p}{,} \PY{n}{j}\PY{p}{:} \PY{k+kt}{int}\PY{p}{)}
\PY{p}{\PYZob{}}
  \PY{n}{isValidProblem}\PY{p}{(}\PY{n}{p}\PY{p}{)} \PY{o}{\PYZam{}\PYZam{}} \PY{l+m+mi}{1} \PY{o}{\PYZlt{}=} \PY{n}{i} \PY{o}{\PYZlt{}=} \PY{n}{p}\PY{p}{.}\PY{n}{n} \PY{o}{\PYZam{}\PYZam{}} \PY{l+m+mi}{1} \PY{o}{\PYZlt{}=} \PY{n}{j} \PY{o}{\PYZlt{}=} \PY{n}{p}\PY{p}{.}\PY{n}{c} 
\PY{p}{\PYZcb{}}
\end{Verbatim}
    este folosit pentru a defini o subproblemă a problemei inițiale.
    \item Predicatul \textit{areValidPartialSolutions}
    \begin{Verbatim}[commandchars=\\\{\}]
\PY{k+kd}{ghost} \PY{k+kd}{predicate} \PY{n+nf}{areValidPartialSolutions}\PY{p}{(}\PY{n}{p}\PY{p}{:} \PY{n}{Problem}\PY{p}{,} \PY{n}{profits}\PY{p}{:} 
    \PY{k+kt}{seq}\PY{o}{\PYZlt{}}\PY{k+kt}{seq}\PY{o}{\PYZlt{}}\PY{k+kt}{int}\PY{o}{\PYZgt{}}\PY{o}{\PYZgt{}}\PY{p}{,} \PY{n}{solutions}\PY{p}{:} \PY{k+kt}{seq}\PY{o}{\PYZlt{}}\PY{k+kt}{seq}\PY{o}{\PYZlt{}}\PY{k+kt}{seq}\PY{o}{\PYZlt{}}\PY{k+kt}{int}\PY{o}{\PYZgt{}}\PY{o}{\PYZgt{}}\PY{o}{\PYZgt{}}\PY{p}{,} 
    \PY{n}{partialProfits}\PY{p}{:} \PY{k+kt}{seq}\PY{o}{\PYZlt{}}\PY{k+kt}{int}\PY{o}{\PYZgt{}}\PY{p}{,} \PY{n}{partialSolutions}\PY{p}{:} \PY{k+kt}{seq}\PY{o}{\PYZlt{}}\PY{k+kt}{seq}\PY{o}{\PYZlt{}}\PY{k+kt}{int}\PY{o}{\PYZgt{}}\PY{o}{\PYZgt{}}\PY{p}{,} 
    \PY{n}{i}\PY{p}{:} \PY{k+kt}{int}\PY{p}{,} \PY{n}{j}\PY{p}{:} \PY{k+kt}{int}\PY{p}{)}    
  \PY{k}{requires} \PY{n}{isValidSubproblem}\PY{p}{(}\PY{n}{p}\PY{p}{,} \PY{n}{i}\PY{p}{,} \PY{n}{j}\PY{p}{)}
\PY{p}{\PYZob{}}
  \PY{o}{|}\PY{n}{partialSolutions}\PY{o}{|} \PY{o}{==} \PY{o}{|}\PY{n}{partialProfits}\PY{o}{|} \PY{o}{==} \PY{n}{j} \PY{o}{\PYZam{}\PYZam{}} 
  \PY{p}{(}\PY{k}{forall} \PY{n}{k} \PY{p}{::} \PY{l+m+mi}{0} \PY{o}{\PYZlt{}=} \PY{n}{k} \PY{o}{\PYZlt{}} \PY{o}{|}\PY{n}{partialSolutions}\PY{o}{|} \PY{o}{==}\PY{o}{\PYZgt{}} 
    \PY{n}{isOptimalPartialSolution}\PY{p}{(}\PY{n}{p}\PY{p}{,} \PY{n}{partialSolutions}\PY{p}{[}\PY{n}{k}\PY{p}{]}\PY{p}{,} \PY{n}{i}\PY{p}{,} \PY{n}{k}\PY{p}{)}\PY{p}{)} \PY{o}{\PYZam{}\PYZam{}} 
  \PY{p}{(}\PY{k}{forall} \PY{n}{k} \PY{p}{::} \PY{l+m+mi}{0} \PY{o}{\PYZlt{}=} \PY{n}{k} \PY{o}{\PYZlt{}} \PY{o}{|}\PY{n}{partialSolutions}\PY{o}{|} \PY{o}{==}\PY{o}{\PYZgt{}} 
    \PY{n}{gain}\PY{p}{(}\PY{n}{p}\PY{p}{,} \PY{n}{partialSolutions}\PY{p}{[}\PY{n}{k}\PY{p}{]}\PY{p}{)} \PY{o}{==} \PY{n}{partialProfits}\PY{p}{[}\PY{n}{k}\PY{p}{]}\PY{p}{)}
\PY{p}{\PYZcb{}}
\end{Verbatim}
    este folosit pentru a valida rezultatele obținute doar pentru subproblemele pasului curent, în funcție de modificările aduse datorate creșterii numărului de obiecte disponibile.
    \item Predicatul \textit{areValidSolutions}
    \begin{Verbatim}[commandchars=\\\{\}]
\PY{k+kd}{ghost} \PY{k+kd}{predicate} \PY{n+nf}{areValidSolutions}\PY{p}{(}\PY{n}{p}\PY{p}{:} \PY{n}{Problem}\PY{p}{,} \PY{n}{profits}\PY{p}{:} 
    \PY{k+kt}{seq}\PY{o}{\PYZlt{}}\PY{k+kt}{seq}\PY{o}{\PYZlt{}}\PY{k+kt}{int}\PY{o}{\PYZgt{}}\PY{o}{\PYZgt{}}\PY{p}{,} \PY{n}{solutions}\PY{p}{:} \PY{k+kt}{seq}\PY{o}{\PYZlt{}}\PY{k+kt}{seq}\PY{o}{\PYZlt{}}\PY{k+kt}{seq}\PY{o}{\PYZlt{}}\PY{k+kt}{int}\PY{o}{\PYZgt{}}\PY{o}{\PYZgt{}}\PY{o}{\PYZgt{}}\PY{p}{,} \PY{n}{i}\PY{p}{:} \PY{k+kt}{int}\PY{p}{)}
  \PY{k}{requires} \PY{n}{isValidSubproblem}\PY{p}{(}\PY{n}{p}\PY{p}{,} \PY{n}{i}\PY{p}{,} \PY{n}{p}\PY{p}{.}\PY{n}{c}\PY{p}{)}
\PY{p}{\PYZob{}} 
  \PY{n}{i} \PY{o}{==} \PY{o}{|}\PY{n}{profits}\PY{o}{|} \PY{o}{==} \PY{o}{|}\PY{n}{solutions}\PY{o}{|} \PY{o}{\PYZam{}\PYZam{}} \PY{p}{(}\PY{k}{forall} \PY{n}{k} \PY{p}{::} \PY{l+m+mi}{0} \PY{o}{\PYZlt{}=} \PY{n}{k} \PY{o}{\PYZlt{}} \PY{n}{i} 
  \PY{o}{==}\PY{o}{\PYZgt{}} \PY{o}{|}\PY{n}{profits}\PY{p}{[}\PY{n}{k}\PY{p}{]}\PY{o}{|} \PY{o}{==} \PY{o}{|}\PY{n}{solutions}\PY{p}{[}\PY{n}{k}\PY{p}{]}\PY{o}{|} \PY{o}{==} \PY{n}{p}\PY{p}{.}\PY{n}{c} \PY{o}{+} \PY{l+m+mi}{1}\PY{p}{)} \PY{o}{\PYZam{}\PYZam{}} 
  \PY{p}{(}\PY{k}{forall} \PY{n}{k} \PY{p}{::} \PY{l+m+mi}{0} \PY{o}{\PYZlt{}=} \PY{n}{k} \PY{o}{\PYZlt{}} \PY{o}{|}\PY{n}{solutions}\PY{o}{|} \PY{o}{==}\PY{o}{\PYZgt{}} 
    \PY{k}{forall} \PY{n}{q} \PY{p}{::} \PY{l+m+mi}{0} \PY{o}{\PYZlt{}=} \PY{n}{q} \PY{o}{\PYZlt{}} \PY{o}{|}\PY{n}{solutions}\PY{p}{[}\PY{n}{k}\PY{p}{]}\PY{o}{|} \PY{o}{==}\PY{o}{\PYZgt{}} 
      \PY{n}{isOptimalPartialSolution}\PY{p}{(}\PY{n}{p}\PY{p}{,} \PY{n}{solutions}\PY{p}{[}\PY{n}{k}\PY{p}{]}\PY{p}{[}\PY{n}{q}\PY{p}{]}\PY{p}{,} \PY{n}{k}\PY{p}{,} \PY{n}{q}\PY{p}{)}\PY{p}{)} \PY{o}{\PYZam{}\PYZam{}} 
  \PY{p}{(}\PY{k}{forall} \PY{n}{k} \PY{p}{::} \PY{l+m+mi}{0} \PY{o}{\PYZlt{}=} \PY{n}{k} \PY{o}{\PYZlt{}} \PY{o}{|}\PY{n}{solutions}\PY{o}{|} \PY{o}{==}\PY{o}{\PYZgt{}} 
    \PY{k}{forall} \PY{n}{q} \PY{p}{::} \PY{l+m+mi}{0} \PY{o}{\PYZlt{}=} \PY{n}{q} \PY{o}{\PYZlt{}} \PY{o}{|}\PY{n}{solutions}\PY{p}{[}\PY{n}{k}\PY{p}{]}\PY{o}{|} \PY{o}{==}\PY{o}{\PYZgt{}} 
      \PY{n}{gain}\PY{p}{(}\PY{n}{p}\PY{p}{,} \PY{n}{solutions}\PY{p}{[}\PY{n}{k}\PY{p}{]}\PY{p}{[}\PY{n}{q}\PY{p}{]}\PY{p}{)} \PY{o}{==} \PY{n}{profits}\PY{p}{[}\PY{n}{k}\PY{p}{]}\PY{p}{[}\PY{n}{q}\PY{p}{]}\PY{p}{)}
\PY{p}{\PYZcb{}}
\end{Verbatim}
    l-am folosit pentru a verifica faptul că soluțiile parțiale ale subproblemelor sunt valide: că au lungime corespunzătoare și aduc cel mai bun profit pentru subproblema pe care o calculează.
\end{enumerate}

\subsection{Funcții}


\section{Leme importante}

-

\section{Probleme întâmpinate}

-