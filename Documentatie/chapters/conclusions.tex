\chapter*{Concluzii} 
\addcontentsline{toc}{chapter}{Concluzii}

\begin{sloppypar}

În cadrul acestei lucrări am implementat și demonstrat corectitudinea unui algoritm  bazat pe tehnica programării dinamice ce rezolvă varianta discretă a problemei rucsacului folosind Dafny, un limbaj de programare dezvoltat pentru verificarea formală a programelor. De asemenea, cu ajutorul lui Dafny am reușit nu doar să demonstrez corectitudinea algoritmului, dar și faptul că acesta construiește la fiecare pas cea mai bună soluție pentru dimensiunea curentă a problemei și resursele disponibile, garantând că soluția finală este cea optimă global. \par
Pe parcursul procesului de dezvoltare am aprofundat modul în care lucrează algoritmul și am înțeles mai bine cum sunt construite soluțiile parțiale. De asemenea, am dobândit o înțelegere mai detaliată asupra limbajului Dafny, în special a modului în care trebuie să definesc specificațiile și constrângerile unei probleme. O dificultate semnificativă pe care am întâmpinat-o a fost formularea și verificarea lemelor pentru care de foarte multe ori am obținut timeout-uri datorită complexității acestora sau a anumitor precondiții lipsă. \par
Pe baza acestei lucrări consider că sunt deschise oportunități spre  îmbunătațire, de exemplu prin optimizarea modului în care sunt memorate soluțiile parțiale. În implementarea curentă este utilizată o întreagă matrice pentru stocarea soluțiilor, însă acest lucru poate fi îmbunătățit prin utilizarea unui vector ce înlocuiește această matrice, deoarece pentru a rezolva fiecare subproblemă în care avem disponibile primele $i$ obiecte avem nevoie doar de soluțiile subproblemelor ce consideră primele $i - 1$ obiecte. \par
De asemenea, ar fi foarte interesant să realizez o comparație între implementarea algoritmului în Dafny și alte implementări în diferite limbaje de verificare formală.

\end{sloppypar}
