\chapter*{Concluzii} 
\addcontentsline{toc}{chapter}{Concluzii}

\begin{sloppypar}

În cadrul acestei lucrări am implementat și demonstrat corectitudinea unui algoritm  bazat pe tehnica programării dinamice ce rezolvă varianta discretă a problemei rucsacului folosind Dafny, un limbaj de programare dezvoltat pentru verificarea formală a programelor. De asemenea, cu ajutorul lui Dafny am reușit nu doar să demonstrez corectitudinea algoritmului, dar și faptul că acesta construiește la fiecare pas cea mai bună soluție pentru dimensiunea curentă a problemei și resursele disponibile, garantând că soluția finală este cea optimă global. \par
Pe parcursul procesului de dezvoltare consider că am învațat foarte multe lucruri noi atât despre modul în care lucrează algoritmul în sine și cum sunt construite soluțiile parțiale, dar cât și despre Dafny ca limbaj, cu modul său specific de a defini specificațiile și constrângerile unei probleme. O dificultate semnificativă pe care am întâmpinat-o a fost formularea și verificarea lemelor pentru care de foarte multe ori am obținut timeout-uri datorită complexității acestora sau a anumitor precondiții lipsă. \par
Timpul de execuție și verificare al programului este influențat semnificativ de mărimea instanței problemei. În funcție de complexitatea acesteia, timpul de verificare poate să difere. O viitoare analiză ar putea fi realizată pe instanțe mai mari ale problemei pentru a analiza eficiența și capacitatea de a suporta un volum mai mare de date. \par
Pe baza acestei lucrări consider că sunt deschise oportunități spre  îmbunătațire, de exemplu prin optimizarea modului în care sunt memorate soluțiile parțiale sau spre extindere, cum ar fi adăugarea unor constrâgeri suplimentare pentru a crește gradul de dificultate. De asemenea, consider că ar fi foarte interesant să realizez o comparație între implementarea algoritmului în Dafny și alte implementări în diferite limbaje de verificare formală.

\end{sloppypar}