\chapter{Reprezentarea soluțiilor}

\begin{sloppypar}

În cadrul problemei rucsacului putem discuta despre ce reprezintă soluția optimă, cât și soluțiile parțiale optime. \par
Pentru a stoca rezultatele corespunzătoare câștigurilor subproblemelor am folosit o secvență de secvențe numită \texttt{profits}, de tip \texttt{seq<seq<int>>}. Deși această variabilă este o secvență de secvențe, ea poate fi privită ca o matrice bidimensională, iar rezultatele pot fi accesate folosind operatorul de indexare similar unui astfel de tablou: $\texttt{profits}[i][j]$, unde $i$ și $j$ reprezită mărimea subproblemei pe care încercăm să o rezolvăm. 
\par
Pe lângă matricea \texttt{profits}, această lucrare se mai bazează și pe o variabilă numită \texttt{solutions} de tip \texttt{seq<seq<seq<int>>>}, care va stoca fiecare secvență binară corespunzătoare profitului stocat în matricea \texttt{profits}. Ea poate fi privită tot ca un tablou multidimensional, iar valorile sunt accesate similar matricei \texttt{profits}.
\par
O \textbf{soluție optimă} reprezintă selecția de obiecte care produce cel mai bun profit format din câștigul fiecărui obiect care poate fi adăugat în rucsac, dar care în același timp nu depășește capacitatea totală a rucsacului. Această soluție vizează problema completă, deținând decizia de includere a fiecărui obiect, iar profitul calculat trebuie să aibă cea mai mare valoare pe care o putem obține pentru instanța primită.  În cazul acestei lucrări, pentru a verifica soluția finală (despre care am discutat în capitolul anterior) am formulat predicatul \texttt{isOptimalSolution} astfel:
\begin{Verbatim}[commandchars=\\\{\}]
\PY{k+kd}{ghost} \PY{k+kd}{predicate} \PY{n+nf}{isOptimalSolution}\PY{p}{(}\PY{n}{p}\PY{p}{:} \PY{n}{Problem}\PY{p}{,} 
    \PY{n}{solution}\PY{p}{:} \PY{n}{Solution}\PY{p}{)}
  \PY{k}{requires} \PY{n}{isValidProblem}\PY{p}{(}\PY{n}{p}\PY{p}{)}
  \PY{k}{requires} \PY{n}{isValidPartialSolution}\PY{p}{(}\PY{n}{p}\PY{p}{,} \PY{n}{solution}\PY{p}{)}
\PY{p}{\PYZob{}}
  \PY{n}{isOptimalPartialSolution}\PY{p}{(}\PY{n}{p}\PY{p}{,} \PY{n}{solution}\PY{p}{,} \PY{n}{p}\PY{p}{.}\PY{n}{n}\PY{p}{,} \PY{n}{p}\PY{p}{.}\PY{n}{c}\PY{p}{)} \PY{o}{\PYZam{}\PYZam{}}
  \PY{k}{forall} \PY{n}{s}\PY{p}{:} \PY{n}{Solution} \PY{p}{::} \PY{p}{(}\PY{p}{(}\PY{p}{(}\PY{n}{isOptimalPartialSolution}\PY{p}{(}
    \PY{n}{p}\PY{p}{,} \PY{n}{s}\PY{p}{,} \PY{n}{p}\PY{p}{.}\PY{n}{n}\PY{p}{,} \PY{n}{p}\PY{p}{.}\PY{n}{c}\PY{p}{)}\PY{p}{)} \PY{o}{==}\PY{o}{\PYZgt{}} \PY{n}{gain}\PY{p}{(}\PY{n}{p}\PY{p}{,} \PY{n}{solution}\PY{p}{)} \PY{o}{\PYZgt{}=} \PY{n}{gain}\PY{p}{(}\PY{n}{p}\PY{p}{,} \PY{n}{s}\PY{p}{)}\PY{p}{)}\PY{p}{)}
\PY{p}{\PYZcb{}}
\end{Verbatim}

\par Cuvântul cheie \texttt{ghost} este folosit pentru a marca faptul că acest predicat este folosit doar în partea de verificare și nu este inclus în codul executabil \cite{leino2021dafny}.
\par Condițiile exprimate folosind clauza \texttt{requires} sunt numite \textbf{precondiții} și reprezintă proprietăți pe care parametrii trebuie să îi respecte înainte de intrarea în corpul unei funcții, a unei metode, leme sau predicat și care vor fi valabile până la ieșirea din acestea \cite{DBLP:series/natosec/KoenigL12}. Sunt esențiale pentru Dafny deoarece dacă acestea nu sunt satisfăcute la momentul apelului, atunci nu mai există nicio garanție privind comportamenul corect al programului.

\par O \textbf{soluție parțială} este o secvență binară, de lungime variabilă care nu depășește numărul de obiecte și care respectă constrângerea legată de capacitatea rucsacului. Predicatul formulat pentru a confirma o astfel de soluție este \texttt{isPartialSolution}:
    \begin{Verbatim}[commandchars=\\\{\}]
\PY{k+kd}{predicate} \PY{n+nf}{isPartialSolution}\PY{p}{(}\PY{n}{p}\PY{p}{:} \PY{n}{Problem}\PY{p}{,} \PY{n}{solution}\PY{p}{:} \PY{n}{Solution}\PY{p}{,} 
        \PY{n}{i}\PY{p}{:} \PY{k+kt}{int}\PY{p}{,} \PY{n}{j}\PY{p}{:} \PY{k+kt}{int}\PY{p}{)}
  \PY{k}{requires} \PY{n}{isValidProblem}\PY{p}{(}\PY{n}{p}\PY{p}{)}
  \PY{k}{requires} \PY{l+m+mi}{0} \PY{o}{\PYZlt{}=} \PY{n}{i} \PY{o}{\PYZlt{}=} \PY{n}{p}\PY{p}{.}\PY{n}{n}
  \PY{k}{requires} \PY{l+m+mi}{0} \PY{o}{\PYZlt{}=} \PY{n}{j} \PY{o}{\PYZlt{}=} \PY{n}{p}\PY{p}{.}\PY{n}{c}
\PY{p}{\PYZob{}}
  \PY{n}{isValidPartialSolution}\PY{p}{(}\PY{n}{p}\PY{p}{,} \PY{n}{solution}\PY{p}{)} \PY{o}{\PYZam{}\PYZam{}} \PY{o}{|}\PY{n}{solution}\PY{o}{|} \PY{o}{==} \PY{n}{i} \PY{o}{\PYZam{}\PYZam{}}
  \PY{n}{weight}\PY{p}{(}\PY{n}{p}\PY{p}{,} \PY{n}{solution}\PY{p}{)} \PY{o}{\PYZlt{}=} \PY{n}{j}
\PY{p}{\PYZcb{}}
\end{Verbatim}
    și depinde de valoarea de adevăr a predicatului \texttt{isValidPartialSolution}:
    \begin{Verbatim}[commandchars=\\\{\}]
\PY{k+kd}{predicate} \PY{n+nf}{isValidPartialSolution}\PY{p}{(}\PY{n}{p}\PY{p}{:} \PY{n}{Problem}\PY{p}{,} \PY{n}{solution}\PY{p}{:} \PY{n}{Solution}\PY{p}{)}
  \PY{k}{requires} \PY{n}{isValidProblem}\PY{p}{(}\PY{n}{p}\PY{p}{)}
\PY{p}{\PYZob{}}
  \PY{n}{hasAllowedValues}\PY{p}{(}\PY{n}{solution}\PY{p}{)} \PY{o}{\PYZam{}\PYZam{}} \PY{o}{|}\PY{n}{solution}\PY{o}{|} \PY{o}{\PYZlt{}=} \PY{n}{p}\PY{p}{.}\PY{n}{n}
\PY{p}{\PYZcb{}}
\end{Verbatim}
\par Cu ajutorul celor două predicate am verificat dacă o soluție este parțială și validă: are doar elemente de 0 și 1, lungimea acesteia nu depășește numărul de obiecte și are o greutate cel mult egală cu capacitatea $j$ a rucsacului. Predicatul apelat \texttt{hasAllowedValues} este definit astfel:
    \begin{Verbatim}[commandchars=\\\{\}]
\PY{k+kd}{predicate} \PY{n+nf}{hasAllowedValues}\PY{p}{(}\PY{n}{solution}\PY{p}{:} \PY{n}{Solution}\PY{p}{)}
\PY{p}{\PYZob{}}
  \PY{k}{forall} \PY{n}{k} \PY{p}{::} \PY{l+m+mi}{0} \PY{o}{\PYZlt{}=} \PY{n}{k} \PY{o}{\PYZlt{}} \PY{o}{|}\PY{n}{solution}\PY{o}{|} \PY{o}{==}\PY{o}{\PYZgt{}} \PY{n}{solution}\PY{p}{[}\PY{n}{k}\PY{p}{]} \PY{o}{==} \PY{l+m+mi}{0} 
        \PY{o}{|}\PY{o}{|} \PY{n}{solution}\PY{p}{[}\PY{n}{k}\PY{p}{]} \PY{o}{==} \PY{l+m+mi}{1}
\PY{p}{\PYZcb{}}
\end{Verbatim}
și asigură că toate elementele unei soluții aparțin exclusiv mulțimii $\{0, 1\}$, unde 0 reprezintă faptul că obiectul nu este inclus în rucsac, iar 1 înseamnă că obiectul este inclus în rucsac.

O \textbf{soluție parțială optimă} reprezintă o secvență de elemente de 0 și 1, al cărei profit este optim pentru un subset de obiecte sau pentru o capacitate parțială a rucsacului. Luând un exemplu prezentat în introducere, o soluție parțială optimă pentru subproblema $(i = 3, j = 8)$ este $[0, 1, 1]$ deoarece aduce cel mai bun profit pentru un subset ce include doar primele trei obiecte, având la dispoziție un rucsac de capacitate maximă 8. Pentru a verifica dacă avem o astfel de soluție am implementat predicatul \texttt{isOptimalPartialSolution}:
    \begin{Verbatim}[commandchars=\\\{\}]
\PY{k+kd}{ghost} \PY{k+kd}{predicate} \PY{n+nf}{isOptimalPartialSolution}\PY{p}{(}\PY{n}{p}\PY{p}{:} \PY{n}{Problem}\PY{p}{,} 
    \PY{n}{solution}\PY{p}{:} \PY{n}{Solution}\PY{p}{,} \PY{n}{i}\PY{p}{:} \PY{k+kt}{int}\PY{p}{,} \PY{n}{j}\PY{p}{:} \PY{k+kt}{int}\PY{p}{)}
  \PY{k}{requires} \PY{n}{isValidProblem}\PY{p}{(}\PY{n}{p}\PY{p}{)}
  \PY{k}{requires} \PY{l+m+mi}{0} \PY{o}{\PYZlt{}=} \PY{n}{i} \PY{o}{\PYZlt{}=} \PY{n}{p}\PY{p}{.}\PY{n}{n}
  \PY{k}{requires} \PY{l+m+mi}{0} \PY{o}{\PYZlt{}=} \PY{n}{j} \PY{o}{\PYZlt{}=} \PY{n}{p}\PY{p}{.}\PY{n}{c}
\PY{p}{\PYZob{}}
  \PY{n}{isPartialSolution}\PY{p}{(}\PY{n}{p}\PY{p}{,} \PY{n}{solution}\PY{p}{,} \PY{n}{i}\PY{p}{,} \PY{n}{j}\PY{p}{)} \PY{o}{\PYZam{}\PYZam{}}
  \PY{k}{forall} \PY{n}{s}\PY{p}{:} \PY{n}{Solution} \PY{p}{::} \PY{p}{(}\PY{n}{isPartialSolution}\PY{p}{(}\PY{n}{p}\PY{p}{,} \PY{n}{s}\PY{p}{,} \PY{n}{i}\PY{p}{,} \PY{n}{j}\PY{p}{)} \PY{o}{\PYZam{}\PYZam{}} 
    \PY{o}{|}\PY{n}{s}\PY{o}{|} \PY{o}{==} \PY{o}{|}\PY{n}{solution}\PY{o}{|} \PY{o}{==}\PY{o}{\PYZgt{}} \PY{n}{gain}\PY{p}{(}\PY{n}{p}\PY{p}{,} \PY{n}{solution}\PY{p}{)} \PY{o}{\PYZgt{}=} \PY{n}{gain}\PY{p}{(}\PY{n}{p}\PY{p}{,} \PY{n}{s}\PY{p}{)}\PY{p}{)}
\PY{p}{\PYZcb{}}
\end{Verbatim}
\par Acest predicat verifică dacă orice altă soluție parțială pentru o subproblemă cu $i$ obiecte și capacitate $j$ a rucsacului va avea câștigul mai mic sau cel mult egal cu cel al soluției noastre. 

\end{sloppypar}